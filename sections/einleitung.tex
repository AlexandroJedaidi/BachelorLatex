\section{Einleitung}
\label{sec:einleitung}
Gewöhnliche Differentialgleichungen lassen sich in der heutigen Zeit in unterschiedlichsten Anwendungen finden.
Es ist beispielsweise möglich, die Bewegungen eines Körpers unter Einfluss der Anziehung anderer Planeten durch
ein System gewöhnlicher Differentialgleichungen darzustellen. Andererseits können durch sogenannte
Lotka-Volterra-Gleichungen, oder auch als Räuber-Beute-Gleichungen bekannt, die Wechselwirkung von Räuber- und
Beutetierpopulation beschrieben werden, was wiederum Anwendung in der Biologie findet. Hierbei fällt auf, dass hohes
Interesse an einer Lösung dieser Systeme besteht, wobei nicht für jede gewöhnliche Differentialgleichung eine Existenz
dieser Lösung garantiert ist.\\
Falls jedoch die Existenz und gegebenfalls sogar die Eindeutigkeit einer Lösung des gegebenen Systems vorliegt, ist es
meistens mit viel Aufwand verbunden, diese Lösung analytisch zu ermittlen. Deshalb haben sich in den letzten
Jahrhunderten die numerische Vorgehensweise zur Lösung gewöhnlicher Differentialgleichungen entwickelt.
Diese Verfahren wurden über die Zeit optimiert und durch den exponentiellen Anstieg an Rechenleistung im 21. Jahrhundert
können heutzutage sogar Systeme hoher Komplexität mithilfe dieser numerischen Verfahren in angemessener Zeit und
brauchbarer Genauigkeit gelöst werden. \\
Mit der rasanten Entwicklung der Computer fanden aber auch Verfahren des maschinellen Lernens zunehmende
Anwendungsbereiche. Allgemein finden die sogenannten neuronalen Netze des maschinellen Lernens ihre Anwendung in
beispielsweise der Bild- oder Spracherkennung. Jedoch ist die Eigenschaft neuronaler Netze, Funktionen darstellen zu
können, nützlich zur Approximation der Lösung einer gewöhnlichen Differentialgleichung. \\
In dieser Arbeit werden wir demnach den Vergleich von numerischen Verfahren mit neuronalen Netzen zur Lösung
gewöhnlicher Differentialgleichungen diskutieren. Zu Beginn werden wir in Kapitel \ref{sec:theorie} Details der
Theorie von gewöhnlichen Differentialgleichungen aufarbeiten. Darauffolgend behandeln wir in Kapitel \ref{sec:numeric}
die Einschritt- und Mehrschrittverfahren, sowie die BDF-Verfahren, welche als numerische Verfahren gelten.
Im Anschluss untersuchen wir eine sogenannte Methode der Lösungspakete mit neuronalen Netzen in
Kapitel \ref{sec:neuralnet} und zuletzt vergleichen wir in Kapitel \ref{sec:verfahrensvergleich} die behandelten
Verfahren anhand drei verschiedenen Anfangswertproblemen mithilfe einiger Fehlermaße. Wir werden auf die Einführung
sowie den Vergleich numerischer Verfahren zur Lösung von Randwertproblemen mit Verfahren des maschinellen Lernens
verzichten.
\newpage

