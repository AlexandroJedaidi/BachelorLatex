\section{Numerischer Lösungsansatz}
Unser Ziel ist es, eine effektive Methode zu verwenden, um eine Lösung eines Anfangswertproblem finden zu können.
Da es nicht immer möglich ist, eine ODE analytisch zu lösen, gibt es sogenannte Ein- und Mehrschrittverfahren, welche
eine Lösung der ODE approximiert.
Der Grundgedanke dieser Verfahren ist es das Zeitintervall $D=[t_0,t_f]$ zu diskretisieren und beginnend mit dem
Anfangswert $t_0$ eine Näherung $u_i \approx x(t_i)$ für $i=0 \dots N$, wobei $t_N=t_f$, berechnet.
Solche Verfahren werden mit Hilfe von Quadraturformeln der numerischen
Integration \cite[Numerische Integration]{walzLexikonMathematik} hergeleitet.\\
Dies basiert darauf, dass die Differentialgleichung auf einem Teilintervall $[t_i, t_{i+1}] \subset D$
integriert wird und nach dem Hauptsatz der Differential- und Integralrechung auf folgende Gleichung gebracht werden kann:
\[
    x(t_{i+1}) - x(t_i) = \int_{t_i}^{t_{i+1}} x^{\prime}(t)dt = \int_{t_i}^{t_{i+1}}f(t, x(t))dt.
\]
Das Integral kann nun mit einer bereits genannten Quadraturformel approximiert werden, welches je nach Wahl der
Quadraturformel ein Einschrittverfahren bildet.
\subsection{Einschrittverfahren}
\begin{definition}
    Sei $D=[t_0,t_0+\alpha]$ ein Zeitintervall und eine Zerlegung von D
    \[
        t_0 < t_1 < \dots < t_N = t_0 + a
    \]
    mit der Schrittweite $h=\frac{a}{N}$, also $D_h=\{t_i=t_0 + ih \text{ für } i=0, \dots,N\}$ eine Intervallzerlegung.
    Ein explizites Einschrittverfahren für das Anfangswertproblem \eqref{eq:5} hat die Form
    \begin{align*}
        u_0 &= x_0\\
        u_{i+1} &= u_i + h\phi(t_i,u_i,h,f), \quad i=0,\dots,N-1, \label{eq:6} \tag{\RNum{6}}
    \end{align*}
    wobei die Lösung $x:D \rightarrow \mathbb{R}^{n}$ auf ganz $D$ existiert.
    Dabei nennt man mit $K \supseteq \mathbb{R}^n$
    $\phi:D_h \times \mathbb{R}^n \times \mathbb{R} \times C(D \times K,\mathbb{R}^n) \rightarrow \mathbb{R}^n$ die
    Inkrementfunktion.
    Implizite Einschrittverfahren haben die Form
    \begin{align*}
        u_0 &= x_0\\
        u_{i+1} &= u_i + h\phi(t_i,u_i,u_{i+1},h,f), \quad i=0,\dots,N-1,
    \end{align*}
    mit $\phi:D_h \times \mathbb{R}^n \times \mathbb{R}^n \times \mathbb{R} \times C(D \times K,\mathbb{R}^n)
    \rightarrow \mathbb{R}^n$.
\end{definition}
\subsubsection{Fehlerdiskussion}
Zur Analyse der Approximationsqualität der Verfahren werden neue Begriffe erläutert.
\begin{definition}[lokaler Diskretisierungsfehler]
    Sei das Anfangswertproblem \eqref{eq:5} mit Lipschitz-stetiger
    rechten Seite und ein explizites Einschrittverfahren \eqref{eq:6} gegeben. Für $\hat{t}\in [t_0,t_0+a]$ und
    $0 < h < t_0 + a - \hat{t}$ ist der lokale Diskretisierungsfehler definiert als
    \[
        \tau(\hat{t}, h) := \frac{x(\hat{t} + h) - u_1(\hat{t})}{h},
    \] wobei $u_1(\hat{t})=x(\hat{t}) + h\phi\left(\hat{t},x(\hat{t}\right),h,f) $ ist die Approximation der exakten
    Lösung nach einem Schritt und mit $x(\hat{t})$ als Startpunkt.\\
    Falls zusätzlich für alle f mit stetiger und beschränkter Ableitung (bis zur Ordnung m) in der x-Variable gilt, dass
    für alle $\hat{t} \in (t_0, t_0+a]$
    \[
        \lim_{h \rightarrow 0 } \tau(\hat{t}, h)=0,
    \] dann nennt man das Einschrittverfahren konsistent (von der Ordnung m).
\end{definition}
\begin{definition}
    Mit $\hat{t} = t_i = t_0+ih, i=1,\dots, N$ ist der globale Diskretisierungsfehler definiert als
    \[
        e(\hat{t}, h) := x(\hat{t}) - u_i.
    \]
    Falls zusätzlich für alle f mit stetiger und beschränkter Ableitung (bis zur Ordnung m) in der x-Variable gilt, dass
    für alle $\hat{t} \in (t_0, t_0+a]$
    \[
        \lim_{h \rightarrow 0 } e(\hat{t}, h)=0,
    \] dann nennt man das Einschrittverfahren konsistent (von der Ordnung m).
\end{definition}
Um eine obere Schranke für den globalen Fehler finden zu können, müssen wir zuerst die diskrete Version der Gronwall
Ungleichung \ref{Satz 6} beweisen.
\begin{satz}[diskrete Gronwallsche Ungleichung]
    Gegeben seien die Folgen $(\alpha)_i,(\beta)_i,(\gamma)_i \geq 0$ mit $i \in \{0,\dots,N\}$ und es gilt
    \[
        \gamma_{i+1} \leq (1 + \alpha_i)\gamma_i + \beta_i \quad \text{für } i=0,\dots,N-1.
    \] Dann gilt
    \[
        \gamma_{i+1} \leq \left( \gamma_0 + \sum_{j=0}^{i}\beta_j \right) e^{\alpha_0 + \dots \alpha_i}  \quad
        \text{für } i=0, \dots, N-1.
    \]
\end{satz}
$Beweis.$ Da $i=1,\dots,N$ bietet sich ein Induktionsbeweis nach i an:
\begin{alignat*}{2}
    (i=0):& \qquad \gamma_1 &&\leq
    \underbrace{(1 + \alpha_0)}_{\leq e^{\alpha_0}}\gamma_0 +
    \underbrace{\beta_0}_{=\beta_0 e^{0} \leq \beta_0 e^{\alpha_0}} \leq (\gamma_0 + \beta_0)e^{\alpha_0}\\
    (\text{IV}):& \qquad
    \gamma_{i} &&\leq \left( \gamma_0 + \sum_{j=0}^{i-1}\beta_j \right) e^{\alpha_0 + \dots \alpha_{i-1}} \\
    (i \rightarrow i+1):& \quad
    \gamma_{i+1} &&\leq (1 + \alpha_{i})\gamma_{i} + \beta_i\\
    & &&\underset{\text{IV}}{\leq} (1 + \alpha_i)
    \left( \gamma_0 + \sum_{j=0}^{i-1}\beta_j \right) e^{\alpha_0 + \dots + \alpha_{i-1}}+\beta_{i}\\
    &  &&\leq e^{\alpha_i} \left( \gamma_0 + \sum_{j=0}^{i-1}\beta_j \right) e^{\alpha_0 + \dots + \alpha_{i-1}}
    + \beta_i e^{\alpha_0 + \dots + \alpha_{i}}\\
    &  &&\leq \left( \gamma_0 +  \sum_{j=0}^{i}\beta_j \right) + e^{\alpha_0 + \dots + \alpha_{i}}
\end{alignat*}

\subsection{Runge-Kutta-Verfahren}
\subsection{Mehrschrittverfahren}
\subsection{Fehlerdisskusion}
\subsubsection{Konsistenz und Konvergenz}
\subsubsection{Stabilität}