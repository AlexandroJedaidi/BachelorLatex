\section{Problemstellung}
In diesem Abschnitt wird die Theorie zu den gewöhnlichen Differentialgleichungen erläutert. Diese ist Grundlage für
das Verständnis der Schlussfolgerungen und Ergebnisse dieser Arbeit.

\subsection{Gewöhnliche Differentialgleichungen und Anfangswertprobleme}
\begin{definition}
    Ein System gewöhnlicher Differentialgleichungen $m-ter$ Ordnung hat die Form
    \[
        x^{(m)} = f(t, x, x^{\prime}, x^{\prime\prime}, ..., x^{(m-1)}) \label{eq:1} \tag{\RNum{1}}
    \]
    mit der gegebenen Funktion
    $
    f : D \times \mathbb{R}^{n} \times \mathbb{R}^{n} \times ... \times \mathbb{R}^{n} \rightarrow \mathbb{R}^{n},
    $
    wobei $D \subseteq \mathbb{R}$ ein Zeitintervall ist. Eine dazugehörige Lösung (falls existent)
    $\hat{x} : D \rightarrow \mathbb{R}$ ist eine $m-mal$ differenzierbare Funktion und erfüllt die Bedingung
    \[
        \hat{x}^{(m)} = f(t, \hat{x},\hat{x}^{\prime},\hat{x}^{\prime\prime}, ...,\hat{x}^{(m-1)}).
    \]
\end{definition}
\begin{definition}
    Ein Anfangswertproblem für eine Differentialgleichung \eqref{eq:1} mit gegebenen Anfangswerten $x_{0,1},
    ...,x_{0,m} \in \mathbb{R}^{n}$ hat die Form
    \[
        x^{(m)} = f(t, x, x^{\prime}, x^{\prime\prime}, ..., x^{(m-1)}),\quad x(t_{0})=x_{0,1},\quad x^{\prime}(t_{0})=x_{0,2} \quad
        ,...,\quad x^{(m-1)}(t_{0})=x_{0,m}. \label{eq:2} \tag{\RNum{2}}
    \]
    Eine Lösung des Problems $\hat{x} : D \rightarrow \mathbb{R}$ muss also zusätzlich zu \eqref{eq:1} auch die
    Anfangswertbedingungen (vgl. \eqref{eq:2}) erfüllen.
\end{definition}
Es ist möglich jede gewöhnliche Differentialgleichung $m-ter$ Ordnung zu einem System gewöhnlicher Differential
gleichungen $erster$ Ordnung umzuwandeln. Dies erleichtert uns in späteren Abschnitten Aussagen über die Existenz
und Eindeutigkeit der Lösung(en) $\hat{x}$ zu treffen.
Betrachte hierzu eine gewöhnliche Differentialgleichung $m-ter$ Ordnung (vgl. \eqref{eq:1}). Diese ist mit Hilfe
der Funktionen $x_{j}:D \rightarrow \mathbb{R}$ für $j \in \{1,\dots,k\}$ äquivalent zu einem System erster Ordnung
mit $m$ Gleichungen:
\begin{align*}
    x_{1}^{\prime}&=x_{2} \\
    x_{2}^{\prime}&=x_{3} \\
    &. \\
    &. \label{eq:3} \tag{\RNum{3}}\\
    &. \\
    x_{m-1}^{\prime}&=x_{m} \\
    x_{m}^{\prime}&=f(t, x_{1}, x_{2}, x_{3}, \ldots, x_{m}). \\
\end{align*}
Für ein AWP \eqref{eq:2} gilt zusätzlich:
\begin{align*}
    x_{1}(t_{0})&=x_{0,1} \\
    x_{2}(t_{0})&=x_{0,2} \\
    &. \\
    &. \label{eq:4} \tag{\RNum{4}}\\
    &. \\
    x_{m-1}(t_{0})&=x_{0,m-1} \\
    x_{m}(t_{0})&=x_{0,m}. \\
\end{align*}

\subsection{Existenz und Eindeutigkeit}
In diesem Abschnitt betrachten wir ein Anfangswertproblem $erster$ Ordnung
\begin{align*}
    x^{\prime}&=f(t,x)\\
    x(t_{0})&=x_{0} \label{eq:5} \tag{\RNum{5}}
\end{align*}
und zeigen, unter welchen Bedingungen der rechten Seite $f(t,x(t))$ eine Lösung existiert und ggf. eindeutig ist.

\subsubsection{Existenz von Lösungen}
Hier betrachten wir einen Satz, welcher zeigt, dass die Stetigkeit der rechten Seite $f$ für die Existenz einer
Lösung ausreicht.
\begin{satz}[Existenzsatz von Peano, quantitative und qualitative Version]
Quantitative Version: Seien
\[
        {\cal G}=\{(t,x) \in \mathbb{R} \times \mathbb{R}^{n}: |t-t_{0}| \leq \alpha, \quad
    \left\lVert x-x_{0} \right\rVert _{2} \leq\beta, \quad \alpha,\beta \geq 0 \}
\]
und $f:{\cal G} \rightarrow \mathbb{R}^{n}$ stetig. Dann besitzt das Anfangswertproblem \eqref{eq:5}
mindestens eine Lösung $\hat{x}$ auf dem Intervall $D=\{t_{0}-a,t_{0}+a\}$, wobei
\[
    a=\min\{\alpha, \frac{\beta}{M}\}, \qquad M= \max_{(t,y)\in {\cal G}}\{\left\lVert f(t,x)\right\rVert_{2} \}
    .
\]\\
Qualitative Version: Seien ${\cal G}\in \mathbb{R} \times \mathbb{R}^{n}$ offen und $f:{\cal G} \rightarrow \mathbb{R}^{n}$ stetig.
Dann besitzt das Anfangswertproblem \eqref{eq:5} für jedes Paar $(t_{0},x_{0}) \in {\cal G}$ mindestens eine
lokale Lösung, d.h., es existiert ein $a=a(t_{0},x_{0}) \geq 0$, sodass das Anfangswertproblem \eqref{eq:5} auf
dem Intervall $[t_{0}-a,t_{0}+a]$ mindestens eine Lösung $\hat{x}$ hat.
\end{satz}
$Beweis$ \cite[52-55]{beckGewohnlicheDifferentialgleichungen2018}

\subsubsection{Eindeutigkeit von Lösungen}
Ähnlich wie im vorherigen Kapitel existiert ein Satz, welcher zeigt, dass eine $lipschitz$-stetige~\cite{LipschitzStetigkeitSerloMathe}
rechte Seite $f$ ausreicht, damit eine eindeutige Lösung für eine
gewöhnliche Differentialgleichung erster Ordnung existiert. Dazu definieren wir zuerst eine Lipschitz-stetige Funktion.
\begin{definition}
    Sei $M \subset \mathbb{R} \times \mathbb{R}^{n}$ und $f: M \rightarrow \mathbb{R}^n$ eine Funktion. Dann heißt genau
    dann f Lipschitz-stetig auf $M$ bezüglich der x-Variable, wenn ein $L>0$ existiert, so dass gilt:
    \[
        \left\lVert f(t,x) - f(t,y) \right\rVert_2 \leq L \left\lVert x -y  \right\rVert_2
    \]
    für alle $(t,x), (t,y) \in M$.
\end{definition}
Nun können wir einen grundlegenden Satz in der Theorie für gewöhnliche Differentialgleichunen betrachten(beweisen
falls ich den beweis noch mache mit Fixpunktsatz davor).
\begin{satz}[Existenzsatz- und Eindeutigkeitssatz von Picard-Lindelöf]
Quantitative Version: Seien
\[
        {\cal G}=\{(t,x) \in \mathbb{R} \times \mathbb{R}^{n}: |t-t_{0}| \leq \alpha, \quad
    \left\lVert x-x_{0} \right\rVert _{2} \leq\beta, \quad \alpha,\beta \geq 0 \},
\]
$(t_{o},x_{0})\in {\cal G}$ und $f:{\cal G} \rightarrow \mathbb{R}^{n}$ stetig und Lipschitz-stetig bzgl $x$.
Dann besitzt das Anfangswertproblem \eqref{eq:5} genau eine Lösung $\hat{x}$ auf dem Intervall
$D=\{t_{0}-a,t_{0}+a\}$, wobei
\[
    a=\min\{\alpha, \frac{\beta}{M}\}, \qquad M= \max_{(t,y)\in {\cal G}}\{\left\lVert f(t,x)\right\rVert_{2} \}
    .
\]\\
Qualitative Version: Seien ${\cal G}\in \mathbb{R} \times \mathbb{R}^{n}$ offen und $f:{\cal G} \rightarrow \mathbb{R}^{n}$ stetig und lokal
Lipschitz-stetig bzgl. $x$ auf $\cal{G}$. Dann besitzt das Anfangswertproblem \eqref{eq:5} für jedes Paar
$(t_{0},x_{0}) \in {\cal G}$ genau eine lokale Lösung, d.h., es existiert ein $a=a(t_{0},x_{0}) \geq 0$, sodass
das Anfangswertproblem \eqref{eq:5} auf dem Intervall $[t_{0}-a,t_{0}+a]$ genau eine Lösung $\hat{x}$ hat.
\end{satz}
$Beweis$ \cite[56-58]{beckGewohnlicheDifferentialgleichungen2018}

\subsection{Abhängigkeit der Lösungen von den Daten}
In dieser Arbeit betrachten in späteren Abschnitten gewöhnliche Differentialgleichungen, die zur Simulation/Vorhersage
natürlicher Systeme genutzt wird, worin es üblich ist, dass Anfangsdaten durch Messfehler oder ähnlicher Fehler
von tatsächlichen Daten abeweichen. Deshalb ist es sinnvoll Aussagen zu betrachen, die zeigen, welche Auswirkungen
kleine Störungen auf die Lösung der Differentialgleichungen haben. In diesem Sektion ist die rechte Seite $f$ stetig und
Lipschitz-stetig bezüglich der x-Variable, sodass wir die Eindeutigkeit der Lösung durch Picard-Lindelöff garantieren.
Der große Vorteil hierfür ist, dass wir keine Maximal- und Minimallösungen betrachten müssen.
Um eine Aussage über die stetige Abhängigkeit der Anfangsdaten teffen zu können, beweisen wir zuerst einen wichtigen
Hilfssatz.
\begin{theorem}[Gronwallsche Ungleichung]
    \label{Satz 5}
    Seien $D=[t_{0}, t_{f}]$ ein Intervall und die stetige, nichtnegative Funktion $u : D \rightarrow \mathbb{R}$
    sowie $a \geq 0, b > 0$ gegeben. Des Weiteren gilt folgende Ungleichung:
    \[
        u(t) \leq \alpha \int_{t_{0}}^{t}u(s)ds + \beta
    \]
    für alle $t \in D$. Dann gilt:
    \[
        u(t) \leq e^{\alpha(t-t_{0})}\beta
    \]
    für alle $t \in D$.
\end{theorem}
$Beweis.$ Definiere zuerst eine Hilfsfunktion
\[
    v(t) := \alpha \int_{t_{0}}^{t} u(s)ds + \beta.
\] Für diese gilt
\[
    v^\prime(t) = \alpha u(t) \leq \alpha v(t)
\] für alle $t \in D$. Daraus folgt
\[
    (e^{-\alpha t}v(t))^\prime = e^{-\alpha t}(v(t)^\prime-\alpha v(t)) \leq 0, \qquad t \in D.
\]
Die Funktion $e^{-\alpha t} v(t)$ ist also monoton fallend, das bedeutet
\[
    e^{-\alpha t} u(t) \leq e^{-\alpha t} v(t) \stackrel{t \geq t_{0}}{\leq} e^{-\alpha t_{0}} v(t_{0}) = e^{-\alpha t_{0}}\beta.
\] Daraus folgt die Behauptung. \qedwhite \\
Außerdem benötigen wir noch folgendes Lemma.
\begin{lemma}
    Sei $T \subset \mathbb{R}^{1 + n}$ offen und $f:T \rightarrow \mathbb{R}$ eine stetige Funktion, die zusätzlich
    Lipschitz-stetig bezüglich der x-Variable ist mit
    \[
        \left\lVert f(t, x) - f(t,y) \right\rVert_{2} \leq L \left\lVert x - y \right\rVert_{2}
    \]
    für alle $(t,x),(t,y) \in T$ mit $L > 0$.
    Ist $\hat{x}$ eine stetig-differenzierbare Funktion auf dem Intervall $D \subset \mathbb{R}$ und eine Lösung des
    Anfangswertproblems \eqref{eq:5} und ist $\hat{x}_a$ eine stetig-differenzierbare Funktion und eine
    Näherungslösung mit $(t,\hat{x}_a(t))\in T$ für alle $t \in D$ und es gilt
    \begin{align*}
        \left\lVert \hat{x}_a^{\prime}(t) - f(t,\hat{x}_a(t)) \right\rVert_{2} &\leq d_e \qquad t \in D,\\
        |t_{0} - \tilde{t_0}| &\leq d_t,\\
        \left\lVert x_0 - \hat{x}_a(\tilde{t_0}) \right\rVert_{2} &\leq d_a\\
    \end{align*}
    ($d_g$ representiert die Störung der rechten Seite, $d_t$ die Störung der Anfangszeit und $d_a$ die Störung
    des Anfangswerts).
    Dann gilt die Abschätzung
    \[
        \left\lVert \hat{x}(t) - \hat{x}_a(t) \right\rVert_{2} \leq
        e^{L|t-t_0|}(d_a + d_t(d_g + \sup_{s \in D} \left\lVert f(s, \hat{x}_a(s)) \right\rVert_2)
        + \frac{d_g}{L}) - \frac{d_g}{L}.
    \]
\end{lemma}
$Beweis.$ Betrachte zuerst die Differenz der Lösung $\hat{x}$ und $\hat{x}_a$ bei $t = t_0$.
\begin{align*}
    \left\lVert \hat{x}(t_0) - \hat{x}_a(t_0) \right\rVert_2 &= \left\lVert \hat{x}_0 -
    \int_{\tilde{t}_0}^{t_0} \hat{x}_a^\prime(s)ds - \hat{x}_a(\tilde{t}_{0}) \right\rVert_2 \\
    & \leq \left\lVert_2 \hat{x}_0 - \hat{x}_a(\tilde{t}_0)\right\rVert
    \left\lVert \int_{\tilde{t}_0}^{t_0} [\hat{x}_a^\prime(s) - f(s, \hat{x}_a(s))] ds \right\rVert_2
    \left\lVert \int_{\tilde{t}_0}^{t_0} f(s,\hat{x}_a(s)) ds \right\rVert_2 \\
    & \leq d_a + d_t(d_g + \sup_{s \in D} \left\lVert f(s,\hat{x}_a(s)) \right\rVert).
\end{align*}
Nun können wir mit Hilfe der Lipschitz-Eigenschaft der rechten Seite $f$ die Differenz für allgemeines
$t \in D , t > t_0$ abschätzen:
\begin{align*}
    \left\lVert \hat{x}(t) - \hat{x}_a(t) \right\rVert_2 &=
    \left\lVert \hat{x}_0 + \int_{t_0}^{t} f(s,\hat{x})ds - \hat{x}_a(t_0) - \int_{t_0}^{t} \hat{x}_a^{\prime}(s) ds \right\rVert_2\\
    &\leq \left\lVert \hat{x}_0 - \hat{x}_a(t_0) \right\rVert_2 +
    \int_{t_0}^{t} [\left\lVert f(s,\hat{x}(s)) - f(s,\hat{x}_a(s)) \right\rVert_2 +
    \left\lVert \hat{x}_a^{\prime}(s) - f(s,\hat{x}_a(s)) \right\rVert_2] ds \\
    &\leq d_a + d_t(d_g + \sup_{s\in D}\left\lVert f(s,\hat{x}_a(s)) \right\rVert_2) +
    \int_{t_0}^{t} [L \left\lVert \hat{x}(s) - \hat{x}_a(s) \right\rVert_2 + d_g] ds.
\end{align*}
Um das gronwallsche Lemma verwenden zu können, setzen wir
$u(t):=\left\lVert \hat{x}(t) - \hat{x}_a(t)\right\rVert_2 + \frac{d_g}{L}$,
\[
    \beta:=d_a + d_t(d_g + \sup_{s\in D}\left\lVert f(s,\hat{x}_a(s)) \right\rVert_2 + \frac{d_g}{L})
\] und $\alpha:=L$.
Offensichtlich gilt
\begin{align*}
    &u(t) \leq \alpha \int_{t_0}^{t} u(s) ds + \beta\\
    \Leftrightarrow & \left\lVert \hat{x}(t) - \hat{x}_a(t)\right\rVert_2 + \frac{d_g}{L} \leq
    L \int_{t_0}^{t} \left[\left\lVert \hat{x}(s) - \hat{x}_a(s)\right\rVert_2 + \frac{d_g}{L}\right] ds + \beta \\
    \Leftrightarrow & \left\lVert \hat{x}(t) - \hat{x}_a(t)\right\rVert_2 \leq
    d_a + d_t(d_g + \sup_{s\in D}\left\lVert f(s,\hat{x}_a(s)) \right\rVert_2) +
    \int_{t_0}^{t} \left[ L \left\lVert \hat{x}(s) - \hat{x}_a(s) \right\rVert_2 + d_g \right] ds - \frac{d_g}{L}
\end{align*}
Also können wir das gronwallsche Lemma anwenden und somit folgt
\[
    \left\lVert \hat{x}(t) - \hat{x}_a(t)\right\rVert_2 + \frac{d_g}{L} \leq
    e^{L(t-t_0)}\left[d_a + d_t(d_g + \sup_{s\in D}\left\lVert f(s,\hat{x}_a(s)) \right\rVert_2 + \frac{d_g}{L})\right].
\]
Der Beweis für $t \in D$ mit $t<t_0$ funktioniert analog. \qedwhite\\
Mit diesem Lemma können wir etwas über die stetige Abhängigkeit der Lösung von der Zeitvariable aussagen.
\begin{satz}
    \label{Satz 7}
    Sei $T \subset \mathbb{R}^{1+n}$ offen und $f:T \rightarrow \mathbb{R}^{n}$ eine stetige Funktion, die
    Lipschitz-stetig bezüglich der x-Variable mit Konstante $L>0$ gegeben. Dann hängt die Lösung $x$ des
    Anfangswertproblems \eqref{eq:5} stetig von den Anfangsdaten $(t_0, x_0) \in T$ und der rechten Seite $f$ ab.
    Darunter versteht man:
    ist eine Lösung $x$ auf einem kompakten Intervall $D \subset \mathbb{R}$, eine Umgebung U des Graphen
    $\{(t,x(t): t \in D)$ in T und ein $\epsilon>0$ gegeben, dann existiert ein $\delta(\epsilon, U, f, D) >0$ in,
    sodass die Lösung $\hat{x}$ des gestörten Anfangswertproblems
    \begin{align*}
        \hat{x}^{\prime} &= \hat{f}(t,\hat{x})\\
        \hat{x}(\hat{t}_0) &= \hat{x}_0 \\
    \end{align*}
    auf ganz D existiert und die Abschätzung
    \[
        \left\lVert x(t) - \hat{x}(t) \right\rVert_2 \leq \epsilon \qquad t \in D
    \]
    erfüllt. Voraussetzungen hierfür sind, dass $\hat{t}_0 \in D$, $(\hat{t}_0, \hat{x}_0) \in T$, $\hat{f}$ stetig auf
    U, Lipschitz-stetig bezüglich der x-Variable und
    \[
        |t_0 - \hat{t}_0| \leq \delta, \quad \left\lVert x_0 - \hat{x}_0 \right\rVert_2 \leq \delta, \quad
        \left\lVert f(t,x) - \hat{f}(t,x) \right\rVert_2 \leq \delta \quad \forall (t,x) \in U
    \] gilt.
\end{satz}
$Beweis.$ \cite[67,68]{beckGewohnlicheDifferentialgleichungen2018}\\
Um die Abhängigkeit der Anfangsdaten $(t_0,x_0)$ formulieren zu können, wird im Folgenden eine Notation eingeführt.
\begin{definition}
    Seien $T \subset \mathbb{R}^{1+n}$ offen und $f:T \rightarrow \mathbb{R}^{n}$ eine stetige Funktion, die
    Lipschitz-stetig bezüglich der x-Variable ist. Die Abbildung
    \[
        (t, t_0, x_0) \mapsto x(t; t_0, x_0)
    \]
    mit $(t_0, x_0)\in T$ und $t \in I_{max}(t_0,x_0) = \left( I^{-}(t_0,x_0), I^{+}(t_0,x_0) \right)$ bezüglich der
    maximalen Lösung des Anfangswertproblems \eqref{eq:5} und dem maximalen Existenzintervall $I_{max}(t_0,x_0)$ heißt
    charakteristische Funktion der gewöhnlichen Differentialgleichung.
    Dabei nennt man 
    \[
        I^{-}(t_0,x_0)=\sup\{ t\in\mathbb{R}: \text{das AWP \eqref{eq:5} ist auf } \left[ t_0, t \right] \text{ lösbar}\}
    \] und
    \[
        I^{+}(t_0,x_0)=\inf\{ t\in\mathbb{R}: \text{das AWP \eqref{eq:5} ist auf } \left[ t, t_0 \right] \text{ lösbar}\}
    \]
    die Lebensdauerfunktion.
\end{definition}
Satz \ref{Satz 7} besagt, dass die charakteristische Funktion in allen Argumenten $(t, t_0, x_0)$ stetig ist und dass
$I^{-}(t_0,x_0)$ bzw. $I^{+}(t_0,x_0)$ ober- bzw. unterhalbstetig sind. Dies bedeutet, dass durch kleine Störungen der
Anfangswerte $(t_0,x_0)$ sich das maximale Existenzintervall nur stetig verkleinern kann.
\begin{satz}
    \label{Satz 9}
    Seien $T \subset \mathbb{R}^{1+n}$ und $f:T \rightarrow \mathbb{R}^{n}$ eine stetige Funktion, die in der x-Variable
    stetig differenzierbar ist. Dann ist die charaktistische Funktion $x(t; t_0, x_0)$ stetig differenzierbar in
    $(t_0, x_0) \in T$ und $t \in I_{max}(t_0, x_0)$.
\end{satz}
$Beweis.$ \cite[69,70]{beckGewohnlicheDifferentialgleichungen2018}\\
\begin{bem}
    Man kann zeigen, dass für $T \subset \mathbb{R}^{1+n}$ und eine stetige Funktion $f:T \rightarrow \mathbb{R}^{n}$,
    die stetig differenzierbar in der x-Variable ist, gilt: $f$ ist lokal Lipschitz-stetig in der x-Variable.
    Daraus folgt, dass in Satz \ref{Satz 9} eine verstärkte Vorraussetzung an die rechte Seite
    im Gegensatz zu Satz \ref{Satz 7} verlangt wird.
\end{bem}