\section{Problemstellung}
In diesem Abschnitt wird die Theorie zu den gewöhnlichen Differentialgleichungen erläutert. Diese ist Grundlage für
das Verständnis der Schlussfolgerungen und Ergebnisse dieser Arbeit.

\subsection{Gewöhnliche Differentialgleichungen und Anfangswertprobleme}
\begin{definition}
    Ein System gewöhnlicher Differentialgleichungen $m-ter$ Ordnung hat die Form
    \begin{align}
        x^{(m)} = f(t, x, x^{\prime}, x^{\prime\prime}, \dots, x^{(m-1)}) \label{eq:1}
    \end{align}
    mit der gegebenen Funktion
    $
    f : D \times \mathbb{R}^{n} \times \mathbb{R}^{n} \times \dots \times \mathbb{R}^{n} \rightarrow \mathbb{R}^{n},
    $
    wobei $D \subseteq \mathbb{R}$ ein Zeitintervall ist. Eine dazugehörige Lösung (falls existent)
    $\hat{x} : D \rightarrow \mathbb{R}$ ist eine $m-mal$ differenzierbare Funktion und erfüllt die Bedingung
    \[
        \hat{x}^{(m)} = f(t, \hat{x},\hat{x}^{\prime},\hat{x}^{\prime\prime}, \dots,\hat{x}^{(m-1)}).
    \]
\end{definition}
\begin{definition}
    Ein Anfangswertproblem für eine Differentialgleichung \eqref{eq:1} mit gegebenen Anfangswerten $x_{0,1},
    \dots,x_{0,m} \in \mathbb{R}^{n}$ hat die Form
    \begin{align}
        x^{(m)} = f(t, x, x^{\prime}, x^{\prime\prime}, \dots, x^{(m-1)}),\quad x(t_{0})=x_{0,1},\quad x^{\prime}(t_{0})=x_{0,2}
        ,\dots,x^{(m-1)}(t_{0})=x_{0,m}. \label{eq:2}
    \end{align}

    Eine Lösung des Problems $\hat{x} : D \rightarrow \mathbb{R}$ muss also zusätzlich zu \eqref{eq:1} auch die
    Anfangswertbedingungen (vgl. \eqref{eq:2}) erfüllen.
\end{definition}
Es ist möglich jede gewöhnliche Differentialgleichung $m-ter$ Ordnung zu einem System gewöhnlicher Differential
gleichungen $erster$ Ordnung umzuwandeln. Dies erleichtert uns in späteren Abschnitten Aussagen über die Existenz
und Eindeutigkeit der Lösung(en) $\hat{x}$ zu treffen.
Betrachte hierzu eine gewöhnliche Differentialgleichung $m-ter$ Ordnung (vgl. \eqref{eq:1}). Diese ist mit Hilfe
der Funktionen $x_{j}:D \rightarrow \mathbb{R}$ für $j \in \{1,\dots,k\}$ äquivalent zu einem System erster Ordnung
mit $m$ Gleichungen:
\begin{align}
    x_{1}^{\prime}&=x_{2} \nonumber \\
    x_{2}^{\prime}&=x_{3} \nonumber \\
    &\vdots\\
    x_{m-1}^{\prime}&=x_{m} \nonumber \\
    x_{m}^{\prime}&=f(t, x_{1}, x_{2}, x_{3}, \ldots, x_{m}). \nonumber \\ \nonumber
\end{align}
Für ein AWP \eqref{eq:2} gilt zusätzlich:
\begin{align}
    x_{1}(t_{0})&=x_{0,1} \nonumber \\
    x_{2}(t_{0})&=x_{0,2} \nonumber \\
    &\vdots \\
    x_{m-1}(t_{0})&=x_{0,m-1} \nonumber \\
    x_{m}(t_{0})&=x_{0,m}. \nonumber \\ \nonumber
\end{align}
\begin{definition}
    Ein System gewöhnlicher Differentialgleichungen \eqref{eq:1} heißt $autonom$, wenn die rechte Seite $f$ nicht
    explizit von der unabhängigen Variable $t$ abhängt. Das bedeutet, es gilt
    \[
        x^{m} = f(x, x^{\prime}, x^{\prime\prime}, \dots, x^{(m-1)})
    \]
    auf ganz $D$.
\end{definition}
In ähnlichem Stil zu der Ordnungsreduktion können wir auch eine nichtautonome Differentialgleichung zu einem autonomen
System gewöhnlicher Differentialgleichungen umwandeln. Dazu führen wir eine neue Variable $z$ ein, für die gilt
\[
    z^{\prime} = 1, \quad z(t_0) = t_0.
\]
Dann ist das Anfangswertproblem \eqref{eq:2} äquivalent zu dem autonomen Anfangswertproblem $m+1$-ter Ordnung
\begin{alignat*}{3}
    z^{\prime} &= 1, \qquad &z(t_0) &= t_0 \\
    x^{\prime} &= f(z,x),\qquad  &x(t_0) &= x_0. \\
\end{alignat*}
Die Lösung hat also die Form $\left[ \begin{matrix}
                                         t\\
                                         x(t)\\
                                    \end{matrix} \right].$

\subsubsection{lineare gewöhnliche Differentialgleichungen}
Wir definieren in diesem Abschnitt noch eine Sonderform einer gewöhnlichen Differentialgleichung.
\begin{definition}
    Ein System gewöhnlicher Differentialgleichungen $m$-ter Ordnung mit der Form
    \begin{align} \label{lin-ode}
        x^{m}(t) = \sum_{i=0}^{m-1} a_k(t) x^{j}(t) + g(t).
    \end{align}
    heißt $lineares$ System gewöhnlicher Differentialgleichungen.
    Dabei sind $a_k : D \rightarrow \mathbb{R}^{n \times n}$ und $g : D \rightarrow \mathbb{R}^n$ stetige Funktionen.\\
    Falls $g \equiv 0$, so nennt man \eqref{lin-ode} homogen, sonst inhomogen.
\end{definition}
Die Lösung einer linearen Differentialgleichung erster Ordnung erfüllt also die Gleichung
\[
    x^{\prime}(t) = A(t)x(t) + g(t),
\]
worin $A:D \rightarrow \mathbb{R}^{n \times n}$ und $g:D \rightarrow \mathbb{R}^n$ stetig sind.
\begin{definition}
    Eine lineare Differentialgleicuhng \eqref{lin-ode} mit $A \in \mathbb{R}^{n \times n}$ und
    $b(t) \in \mathbb{R}^n$ heißt $steif$, wenn
    \begin{align}
        1.& \quad Re(\lambda_j(A)) < 0, \qquad \text{ für alle Eigenwerte } \lambda_j(A) \text{ von A, }\\
        2.& \quad \min\limits_{1 \leq j \leq n} Re(\lambda_j(A)) \ll \max\limits_{1 \leq j \leq n} Re(\lambda_j(A)).
    \end{align}
\end{definition}
Wir werden später ein Verfahren betrachten, welches sich speziell dafür eignet, eine steife gewöhnliche
Differentialgleichung zu lösen.

\subsection{Existenz und Eindeutigkeit}
In diesem Abschnitt betrachten wir ein Anfangswertproblem $erster$ Ordnung
\begin{align}
    x^{\prime}&=f(t,x) \nonumber \\
    x(t_{0})&=x_{0} \label{eq:5}
\end{align}
und zeigen, unter welchen Bedingungen der rechten Seite $f(t,x(t))$ eine Lösung existiert und ggf. eindeutig ist.

\subsubsection{Existenz von Lösungen}
Hier beweisen wir einen Satz, welcher zeigt, dass die Stetigkeit der rechten Seite $f$ für die Existenz einer
Lösung ausreicht. Dazu benötigen wir noch einen Satz aus der Funktionalanalysis.
\begin{satz}[Fixpunktsatz von Schauder, 2. Version]
    Sei M eine nichtleere, abgeschlossene und konvexe Teilmenge eines Banachraums $X$ und $A:M \rightarrow M$ stetig.
    Dann besitzt $A$ wenigstens einen Fixpunkt $x$, sofern die Bildmenge $A(M)$ relativ kompakt ist.
\end{satz}
$Beweis.$ \cite[13,14]{sundermeierFixpunktsatzSchauder}
\begin{satz}[Existenzsatz von Peano, quantitative und qualitative Version]
    Quantitative Version: Seien
    \[
        {\cal G}=\{(t,x) \in \mathbb{R} \times \mathbb{R}^{n}: |t-t_{0}| \leq \alpha, \quad
        \left\lVert x-x_{0} \right\rVert _{2} \leq\beta, \quad \alpha,\beta > 0 \}
    \]
    und $f:{\cal G} \rightarrow \mathbb{R}^{n}$ stetig. Dann besitzt das Anfangswertproblem \eqref{eq:5}
    mindestens eine Lösung $\hat{x}$ auf dem Intervall $D=\{t_{0}-a,t_{0}+a\}$, wobei
    \[
        a=\min\{\alpha, \frac{\beta}{M}\}, \qquad M= \max_{(t,y)\in {\cal G}}\left\lVert f(t,x)\right\rVert_{2}.
    \]\\
    Qualitative Version: Seien ${\cal G}\in \mathbb{R} \times \mathbb{R}^{n}$ offen und $f:{\cal G} \rightarrow \mathbb{R}^{n}$ stetig.
    Dann besitzt das Anfangswertproblem \eqref{eq:5} für jedes Paar $(t_{0},x_{0}) \in {\cal G}$ mindestens eine
    lokale Lösung, d.h., es existiert ein $a=a(t_{0},x_{0}) \geq 0$, sodass das Anfangswertproblem \eqref{eq:5} auf
    dem Intervall $[t_{0}-a,t_{0}+a]$ mindestens eine Lösung $\hat{x}$ hat.
\end{satz}
$Beweis.$ (Quantitative Lösung) Sei $J\subset D$ abgeschlossen mit $t_0 \in J$ und $\hat{x}(t)$ eine Lösung von \eqref{eq:5}.
Da die rechte Seite $f$ auf J stetig ist, gilt nach dem Hauptsatz der Differential- und Integralrechnung:
\[
    \hat{x}(t) = \hat{x}_0 + \int_{t_0}^{t} f(s, \hat{x}(s)) ds \quad \text{für alle} \quad t \in J.
\]
Außerdem gilt für jede auf J stetige Funktion $\hat{x}(t)$, die obige Gleichung erfüllt, dass sie auf J differenzierbar
ist. Dies bedeutet, dass eine auf J stetige Funktion $\hat{x}(t)$ löst genau dann das Anfangswertproblem \eqref{eq:5},
wenn es die obige Integralgleichung erfüllt.
für eine auf $J$ stetige Funktion $\hat{x}(t)$ gilt also:
\[
    \hat{x}(t) \text{ löst das AWP \eqref{eq:5}}  \Leftrightarrow \hat{x}(t) \text{ erfüllt obige Gleichung.}
\]
Betrachte nun die Menge $K := \{ x \in C(D): \left\lVert x(t) - x_0 \right\rVert \leq \beta \text{ für alle } t \in D \}$,
wobei $C(D)=\{f: f(t) \text{ ist stetig für alle } t \in D \}$. K ist offensichtlich nicht leer (Betrachte bspw. die
konstante Funktion $x \equiv x_0$). Jedem $x \in K$ wird nun eine Funktion $Ax \in C(D)$ zugewiesen mit der Definition
\[
    (Ax)(t) := x_0 + \int_{t_0}^{t} f(s, x(s)) ds \quad \text{ für alle } \quad t \in D.
\]
Also löst $Ax$ nach obiger Äquivalenz das Anfangswertproblem auf $D$. Außerdem gilt:
\begin{align*}
    \left\lVert (Ax)(t) - x_0 \right\rVert &= \left\lVert \int_{t_0}^{t} f(s, x(s)) ds \right\rVert \\
    &\leq \int_{t_0}^{t} \left\lVert f(s, x(s)) \right\rVert ds \\
    &\leq |t - t_0| \max_{(t,y)\in {\cal G}}\left\lVert f(t,x)\right\rVert_{2} \\
    &\leq |t_0 + a - t_0| M \\
    &\leq \frac{\beta}{M}M = \beta \quad \text{ für alle } t \in D
\end{align*}
Das bedeutet, dass $Ax \in K$, bzw. $AK = K$. Nun können wir die hergeleitete Äquivalenz umformulieren:
Für eine Funktion $\hat{x} \in K$ gilt
\[
    \hat{x} \text{ löst auf D das Anfangswertproblem } \eqref{eq:5} \Leftrightarrow \hat{x}
    \text{ ist ein Fixpunkt der Abbildung } A:K \rightarrow K.
\]
Dazu nutzen wir den Fixpunktsatz von Schauder. Betrachte den Banachraum $C(D)$ mit der Maximumsnorm
\[
    \left\lVert x \right\rVert_{\infty}:=\max_{t \in D} |x(t)|.
\]
Sei $(x_n)_{n \in \mathbb{N}} \subset K$ eine konvergente Folge mit Grenzwert $x \in C(D)$. Dann ist
\[
    \left\lVert x - x_0 \right\rVert = \left\lVert \lim_{n \rightarrow \infty}(x_n) - x_0 \right\rVert
    = \lim_{n \rightarrow \infty} \left\lVert (x_n) - x_0 \right\rVert \leq \lim_{n \rightarrow \infty} \beta = \beta.
\]
Also ist $x \in K$ und somit ist K abgeschlossen. Sei $x,y \in K $ bel. und betrachte $v:=\lambda x + (1-\lambda)y$
mit $0 \leq \lambda \leq 1$. Dann ist $v \in K$, da
\begin{align*}
    \left\lVert v(t) - v_0 \right\rVert
    &= \left\lVert \lambda x(t) + (1-\lambda)y(t) - \lambda x(t_0) - (1-\lambda)y(t_0) \right\rVert \\
    &= \lambda \left\lVert x(t) - x_0 \right\rVert + (1 - \lambda) \left\lVert y(t) - y_0 \right\rVert \\
    &\leq \beta (\lambda + 1 - \lambda) = \beta.
\end{align*}
Das heißt, K ist außerdem konvex. Die Stetigkeit der Abbildung $A:K \rightarrow K$ lässt sich wie folgt zeigen.
Wegen der gleichmäßigen Stetigkeit von f auf ${\cal G}$ können wir ein $\delta > 0$ so bestimmen, sodass gilt:
\[
\left\lVert f(t,x) - f(t,y) \right\rVert < \frac{\epsilon}{a} \text{ für } \left\lVert x - y \right\rVert < \delta.
\]
Betrachte nun $x, y \in K$ mit $\left\lVert x - y \right\rVert_{\infty}$ also auch $\left\lVert x(t) - y(t) \right\rVert$
für alle $t \in D$, so ist für $t \in D$ immer $\left\lVert f(t,x(t)) - f(t,y(t)) \right\rVert < \frac{\epsilon}{a}$ und
es folgt
\[
    \left\lVert (Ax)(t) - (Ay)(t) \right\rVert \leq |t - t_0| \frac{\epsilon}{a} = \epsilon \text{ für jedes } t \in D,
\]
also auch $\left\lVert Ax - Ay \right\rVert_{\infty} \leq \epsilon$. Es bleibt jetzt nur noch zu zeigen, dass die
Bildmenge $A(K)$ relativ kompakt ist. Sei hierfür $x \in K$ beliebig, dann ist
\[
    \left\lVert (Ax)(t) \right\rVert \leq \left\lVert x_0 \right\rVert + aM \quad \text{ für jedes } \quad t \in D
\] und
\[
    \left\lVert (Ax)(t_1) - (Ax)(t_2) \right\rVert \leq |t_1 - t_2| M \quad \text{ für alle } \quad t_1,t_2 \in D.
\]
Somit ist $A(K)$ punktweise beschränkt und gleichgradig stetig auf dem kompakten Intervall D. Nach dem Satz von
Arzela-Ascoli \cite[49]{beckGewohnlicheDifferentialgleichungen2018} hat also jede Folge $(Ax)_n\subset A(K)$ eine
gleichmäßig konvergente Teilfolge. Dies gilt offensichtlich auch für die Maximimsnorm wodurch $A(K)$
relativ kompakt ist. \qedwhite\\

\subsubsection{Eindeutigkeit von Lösungen}
Ähnlich wie im vorherigen Kapitel existiert ein Satz, welcher zeigt, dass eine $Lipschitz$-stetige
rechte Seite $f$ ausreicht, damit eine eindeutige Lösung für eine
gewöhnliche Differentialgleichung erster Ordnung existiert. Dazu definieren wir zuerst eine Lipschitz-stetige Funktion.
\begin{definition}
    Sei $M \subset \mathbb{R} \times \mathbb{R}^{n}$ und $f: M \rightarrow \mathbb{R}^n$ eine Funktion. Dann heißt genau
    dann f Lipschitz-stetig auf $M$ bezüglich der x-Variable, wenn ein $L>0$ existiert, sodass gilt:
    \[
        \left\lVert f(t,x) - f(t,y) \right\rVert \leq L \left\lVert x - y  \right\rVert
    \]
    für alle $(t,x), (t,y) \in M$.
\end{definition}
Außerdem benötigen wir für den Beweis noch den Fixpunktsatz von Weissinger.
\begin{satz}[Fixpunktsatz von Weissinger]
    Sei $\left( B, \left\lVert . \right\rVert \right) $ ein Banachraum und $U \subset B$ abgeschlossen und nichtleer.
    Ferner ist $\sum_{j=1}^{\infty} \alpha_j$ eine konvergente Reihe positiver Zahlen, also
    $\sum_{j=1}^{\infty} \alpha_j < \infty $, $\alpha_j \geq 0$, sowie $A:U \rightarrow U$ eine Selbstabbildung, für die
    \[
        \left\lVert A^{j}u - A^{j}v \right\rVert \leq \alpha_j \left\lVert u - v \right\rVert \quad \text{ für alle }
        \quad u,v \in U \quad \text{ und } \quad j \in \mathbb{N}
    \]
    gilt. Dann besitzt $A$ genau einen Fixpunkt in $U$, nämlich
    \[
        u = \lim_{n\rightarrow \infty} A^{n}u_0
    \]
    mit beliebigem $u_0 \in U$. $u$ ist also der Grenzwert der rekursiven Folge $u_j:=Au_{j-1}$ für $j \geq 1$
    mit $u_0 \in U$ beliebig. Außerdem gilt die Fehlerabschätzung
    \[
        \left\lVert u - u_n \right\rVert \leq \sum_{j=n}^{\infty} \alpha_j \left\lVert u_1 - u_0 \right\rVert.
    \]
\end{satz}
$Beweis.$ \cite[139]{harroheuserGewohnlicheDifferentialgleichungen}\\
%Es gilt
%\[
%    \left\lVert u_{j+1}-u_j \right\rVert = \left\lVert A^j u_1 - A^j u_0 \right\rVert \leq
%    \alpha_j \left\lVert u_1 - u_0 \right\rVert
%\] also können wir folgern
%\[
%    \left\lVert u_{j+k} - u_j \right\rVert \leq \left\lVert u_{j+k-1} - u_{j+k-2} \right\rVert + \dots +
%    \left\lVert u_{j+1} - u_j \right\rVert
%    \leq \underbrace{\left( \alpha_{j+k-1} + \dots + \alpha_{j} \right)}_{\rightarrow 0 \text{ für } j,k
%    \rightarrow\infty} \left\lVert u_1 - u_0\right\rVert
%\]
Nun können wir einen grundlegenden Satz in der Theorie für gewöhnliche Differentialgleichungen beweisen.
\begin{satz}[Existenzsatz- und Eindeutigkeitssatz von Picard-Lindelöf]
Quantitative Version: Seien
\[
        {\cal G}=\{(t,x) \in \mathbb{R} \times \mathbb{R}^{n}: |t-t_{0}| \leq \alpha, \quad
    \left\lVert x-x_{0} \right\rVert _{2} \leq\beta, \quad \alpha,\beta \geq 0 \},
\]
$(t_{o},x_{0})\in {\cal G}$ und $f:{\cal G} \rightarrow \mathbb{R}^{n}$ stetig und Lipschitz-stetig bzgl $x$.
Dann besitzt das Anfangswertproblem \eqref{eq:5} genau eine Lösung $\hat{x}$ auf dem Intervall
$D=\{t_{0}-a,t_{0}+a\}$, wobei
\[
    a=\min\{\alpha, \frac{\beta}{M}\}, \qquad M= \max_{(t,y)\in {\cal G}}\left\lVert f(t,x)\right\rVert_{2} .
\]\\
Qualitative Version: Seien ${\cal G}\in \mathbb{R} \times \mathbb{R}^{n}$ offen und $f:{\cal G} \rightarrow \mathbb{R}^{n}$ stetig und lokal
Lipschitz-stetig bzgl. $x$ auf $\cal{G}$. Dann besitzt das Anfangswertproblem \eqref{eq:5} für jedes Paar
$(t_{0},x_{0}) \in {\cal G}$ genau eine lokale Lösung, d.h., es existiert ein $a=a(t_{0},x_{0}) \geq 0$, sodass
das Anfangswertproblem \eqref{eq:5} auf dem Intervall $[t_{0}-a,t_{0}+a]$ genau eine Lösung $\hat{x}$ hat.
\end{satz}
! TODO: Ungleichung beweisen !
$Beweis$ Nun ist $f$ Lipschitz-stetig, also setzen wir $B=C(D)$ mit der Maximumsnorm, $U=K$ und $A:K \rightarrow K$
aus dem Beweis von Peano. Aus
\[
    \left\lVert (A^j u)(t) - (A^j v)(t) \right\rVert \leq \frac{|t-t_0|^j}{j!} L^j\left\lVert u - v \right\rVert_{\infty}
\]
für alle $j \in \mathbb{N}$ und $t \in D$, folgt direkt
\[
    \left\lVert A^j u - A^j v \right\rVert_{\infty} \leq \frac{(aL)^j}{j!} \left\lVert u - v \right\rVert_{\infty}.
\]
Die Reihe $\sum_{j=1}^{\infty} \frac{(aL)^j}{j!} $ ist nach dem Quotientenkriterium offensichtlich konvergent. Also
lässt sich der Fixpunktsatz von Weissinger anwenden. Der daraus gewonnene Fixpunkt ist eindeutig und mit ähnlicher
Äuqivalenz zu dem Beweis von Peano folgt, dass der Fixpunkt die gesuchte eindeutige Lösung ist. \qedwhite

\subsection{Abhängigkeit der Lösungen von den Daten}
In späteren Abschnitten werden wir gewöhnliche Differentialgleichungen betrachten, die zur Simulation/Vorhersage
natürlicher Systeme genutzt werden. Darin ist es üblich, dass Anfangsdaten durch Messfehler oder unüblicher Verhalten
von tatsächlichen Daten abweichen. Deshalb ist es sinnvoll Aussagen zu betrachten, die zeigen, welche Auswirkungen
kleine Störungen auf die Lösung der Differentialgleichungen haben. In diesem Sektion ist die rechte Seite $f$ stetig und
Lipschitz-stetig bezüglich der x-Variable, sodass wir die Eindeutigkeit der Lösung durch Picard-Lindelöff garantieren.
Der große Vorteil hierfür ist, dass wir keine Maximal- und Minimallösungen betrachten müssen.
Um eine Aussage über die stetige Abhängigkeit der Anfangsdaten treffen zu können, beweisen wir zuerst einen wichtigen
Hilfssatz.
\begin{theorem}[Gronwallsche Ungleichung]
    \label{Satz-gronwall}
    Seien $D=[t_{0}, t_{f}]$ ein Intervall und die stetige, nichtnegative Funktion $u : D \rightarrow \mathbb{R}$
    sowie $a \geq 0, b > 0$ gegeben. Des Weiteren gilt folgende Ungleichung:
    \[
        u(t) \leq \alpha \int_{t_{0}}^{t}u(s)ds + \beta
    \]
    für alle $t \in D$. Dann gilt:
    \[
        u(t) \leq e^{\alpha(t-t_{0})}\beta
    \]
    für alle $t \in D$.
\end{theorem}
$Beweis.$ Definiere zuerst eine Hilfsfunktion
\[
    v(t) := \alpha \int_{t_{0}}^{t} u(s)ds + \beta.
\] Für diese gilt
\[
    v^\prime(t) = \alpha u(t) \leq \alpha v(t)
\] für alle $t \in D$. Daraus folgt
\[
    (e^{-\alpha t}v(t))^\prime = e^{-\alpha t}(v(t)^\prime-\alpha v(t)) \leq 0, \qquad t \in D.
\]
Die Funktion $e^{-\alpha t} v(t)$ ist also monoton fallend, das bedeutet
\[
    e^{-\alpha t} u(t) \leq e^{-\alpha t} v(t) \stackrel{t \geq t_{0}}{\leq} e^{-\alpha t_{0}} v(t_{0}) = e^{-\alpha t_{0}}\beta.
\] Daraus folgt die Behauptung. \qedwhite \\
Außerdem benötigen wir noch folgendes Lemma.
\begin{lemma}
    Sei $T \subset \mathbb{R}^{1 + n}$ offen und $f:T \rightarrow \mathbb{R}$ eine stetige Funktion, die zusätzlich
    Lipschitz-stetig bezüglich der x-Variable ist mit
    \[
        \left\lVert f(t, x) - f(t,y) \right\rVert_{2} \leq L \left\lVert x - y \right\rVert_{2}
    \]
    für alle $(t,x),(t,y) \in T$ mit $L > 0$.
    Ist $\hat{x}$ eine stetig-differenzierbare Funktion auf dem Intervall $D \subset \mathbb{R}$ und eine Lösung des
    Anfangswertproblems \eqref{eq:5} und ist $\hat{x}_a$ eine stetig-differenzierbare Funktion und eine
    Näherungslösung mit $(t,\hat{x}_a(t))\in T$ für alle $t \in D$ und es gilt
    \begin{align*}
        \left\lVert \hat{x}_a^{\prime}(t) - f(t,\hat{x}_a(t)) \right\rVert_{2} &\leq d_e \qquad t \in D,\\
        |t_{0} - \tilde{t_0}| &\leq d_t,\\
        \left\lVert x_0 - \hat{x}_a(\tilde{t_0}) \right\rVert_{2} &\leq d_a\\
    \end{align*}
    ($d_g$ representiert die Störung der rechten Seite, $d_t$ die Störung der Anfangszeit und $d_a$ die Störung
    des Anfangswerts).
    Dann gilt die Abschätzung
    \[
        \left\lVert \hat{x}(t) - \hat{x}_a(t) \right\rVert_{2} \leq
        e^{L|t-t_0|}(d_a + d_t(d_g + \sup_{s \in D} \left\lVert f(s, \hat{x}_a(s)) \right\rVert)
        + \frac{d_g}{L}) - \frac{d_g}{L}.
    \]
\end{lemma}
$Beweis.$ Betrachte zuerst die Differenz der Lösung $\hat{x}$ und $\hat{x}_a$ bei $t = t_0$.
\begin{align*}
    \left\lVert \hat{x}(t_0) - \hat{x}_a(t_0) \right\rVert &= \left\lVert \hat{x}_0 -
    \int_{\tilde{t}_0}^{t_0} \hat{x}_a^\prime(s)ds - \hat{x}_a(\tilde{t}_{0}) \right\rVert \\
    & \leq \left\lVert \hat{x}_0 - \hat{x}_a(\tilde{t}_0)\right\rVert
    \left\lVert \int_{\tilde{t}_0}^{t_0} [\hat{x}_a^\prime(s) - f(s, \hat{x}_a(s))] ds \right\rVert
    \left\lVert \int_{\tilde{t}_0}^{t_0} f(s,\hat{x}_a(s)) ds \right\rVert \\
    & \leq d_a + d_t(d_g + \sup_{s \in D} \left\lVert f(s,\hat{x}_a(s)) \right\rVert).
\end{align*}
Nun können wir mit Hilfe der Lipschitz-Eigenschaft der rechten Seite $f$ die Differenz für allgemeines
$t \in D , t > t_0$ abschätzen:
\begin{align*}
    \left\lVert \hat{x}(t) - \hat{x}_a(t) \right\rVert &=
    \left\lVert \hat{x}_0 + \int_{t_0}^{t} f(s,\hat{x})ds - \hat{x}_a(t_0) - \int_{t_0}^{t} \hat{x}_a^{\prime}(s) ds \right\rVert\\
    &\leq \left\lVert \hat{x}_0 - \hat{x}_a(t_0) \right\rVert +
    \int_{t_0}^{t} [\left\lVert f(s,\hat{x}(s)) - f(s,\hat{x}_a(s)) \right\rVert +
    \left\lVert \hat{x}_a^{\prime}(s) - f(s,\hat{x}_a(s)) \right\rVert] ds \\
    &\leq d_a + d_t(d_g + \sup_{s\in D}\left\lVert f(s,\hat{x}_a(s)) \right\rVert) +
    \int_{t_0}^{t} [L \left\lVert \hat{x}(s) - \hat{x}_a(s) \right\rVert + d_g] ds.
\end{align*}
Um das gronwallsche Lemma verwenden zu können, setzen wir
$u(t):=\left\lVert \hat{x}(t) - \hat{x}_a(t)\right\rVert + \frac{d_g}{L}$,
\[
    \beta:=d_a + d_t(d_g + \sup_{s\in D}\left\lVert f(s,\hat{x}_a(s)) \right\rVert + \frac{d_g}{L})
\] und $\alpha:=L$.
Offensichtlich gilt
\begin{align*}
    &u(t) \leq \alpha \int_{t_0}^{t} u(s) ds + \beta\\
    \Leftrightarrow & \left\lVert \hat{x}(t) - \hat{x}_a(t)\right\rVert + \frac{d_g}{L} \leq
    L \int_{t_0}^{t} \left[\left\lVert \hat{x}(s) - \hat{x}_a(s)\right\rVert + \frac{d_g}{L}\right] ds + \beta \\
    \Leftrightarrow & \left\lVert \hat{x}(t) - \hat{x}_a(t)\right\rVert \leq
    d_a + d_t(d_g + \sup_{s\in D}\left\lVert f(s,\hat{x}_a(s)) \right\rVert) +
    \int_{t_0}^{t} \left[ L \left\lVert \hat{x}(s) - \hat{x}_a(s) \right\rVert + d_g \right] ds - \frac{d_g}{L}
\end{align*}
Also können wir das gronwallsche Lemma anwenden und somit folgt
\[
    \left\lVert \hat{x}(t) - \hat{x}_a(t)\right\rVert + \frac{d_g}{L} \leq
    e^{L(t-t_0)}\left[d_a + d_t(d_g + \sup_{s\in D}\left\lVert f(s,\hat{x}_a(s)) \right\rVert + \frac{d_g}{L})\right].
\]
Der Beweis für $t \in D$ mit $t<t_0$ funktioniert analog. \qedwhite\\
Mit diesem Lemma können wir etwas über die stetige Abhängigkeit der Lösung von der Zeitvariable aussagen.
\begin{satz}
    \label{Satz-stet-abh}
    Sei $T \subset \mathbb{R}^{1+n}$ offen und $f:T \rightarrow \mathbb{R}^{n}$ eine stetige Funktion, die
    Lipschitz-stetig bezüglich der x-Variable mit Konstante $L>0$ gegeben. Dann hängt die Lösung $x$ des
    Anfangswertproblems \eqref{eq:5} stetig von den Anfangsdaten $(t_0, x_0) \in T$ und der rechten Seite $f$ ab.
    Darunter versteht man:
    ist eine Lösung $x$ auf einem kompakten Intervall $D \subset \mathbb{R}$, eine Umgebung U des Graphen
    $\{(t,x(t): t \in D)$ in T und ein $\epsilon>0$ gegeben, dann existiert ein $\delta(\epsilon, U, f, D) >0$ in,
    sodass die Lösung $\hat{x}$ des gestörten Anfangswertproblems
    \begin{align*}
        \hat{x}^{\prime} &= \hat{f}(t,\hat{x})\\
        \hat{x}(\hat{t}_0) &= \hat{x}_0 \\
    \end{align*}
    auf ganz D existiert und die Abschätzung
    \[
        \left\lVert x(t) - \hat{x}(t) \right\rVert \leq \epsilon \qquad t \in D
    \]
    erfüllt. Voraussetzungen hierfür sind, dass $\hat{t}_0 \in D$, $(\hat{t}_0, \hat{x}_0) \in T$, $\hat{f}$ stetig auf
    U, Lipschitz-stetig bezüglich der x-Variable und
    \[
        |t_0 - \hat{t}_0| \leq \delta, \quad \left\lVert x_0 - \hat{x}_0 \right\rVert \leq \delta, \quad
        \left\lVert f(t,x) - \hat{f}(t,x) \right\rVert \leq \delta \quad \forall (t,x) \in U
    \] gilt.
\end{satz}
$Beweis.$ \cite[67,68]{beckGewohnlicheDifferentialgleichungen2018}\\
Um die Abhängigkeit der Anfangsdaten $(t_0,x_0)$ formulieren zu können, wird im Folgenden eine Notation eingeführt.
\begin{definition}
    Seien $T \subset \mathbb{R}^{1+n}$ offen und $f:T \rightarrow \mathbb{R}^{n}$ eine stetige Funktion, die
    Lipschitz-stetig bezüglich der x-Variable ist. Die Abbildung
    \[
        (t, t_0, x_0) \mapsto x(t; t_0, x_0)
    \]
    mit $(t_0, x_0)\in T$ und $t \in I_{max}(t_0,x_0) = \left( I^{-}(t_0,x_0), I^{+}(t_0,x_0) \right)$ bezüglich der
    maximalen Lösung des Anfangswertproblems \eqref{eq:5} und dem maximalen Existenzintervall $I_{max}(t_0,x_0)$ heißt
    charakteristische Funktion der gewöhnlichen Differentialgleichung.
    Dabei nennt man 
    \[
        I^{-}(t_0,x_0)=\sup\{ t\in\mathbb{R}: \text{das AWP \eqref{eq:5} ist auf } \left[ t_0, t \right] \text{ lösbar}\}
    \] und
    \[
        I^{+}(t_0,x_0)=\inf\{ t\in\mathbb{R}: \text{das AWP \eqref{eq:5} ist auf } \left[ t, t_0 \right] \text{ lösbar}\}
    \]
    die Lebensdauerfunktion.
\end{definition}
Satz \ref{Satz-stet-abh} besagt, dass die charakteristische Funktion in allen Argumenten $(t, t_0, x_0)$ stetig ist und
dass $I^{-}(t_0,x_0)$ bzw. $I^{+}(t_0,x_0)$ ober- bzw. unterhalbstetig sind. Dies bedeutet, dass durch kleine Störungen der
Anfangswerte $(t_0,x_0)$ sich das maximale Existenzintervall nur stetig verkleinern kann.
\begin{satz}
    \label{Satz-diff-abh}
    Seien $T \subset \mathbb{R}^{1+n}$ und $f:T \rightarrow \mathbb{R}^{n}$ eine stetige Funktion, die in der x-Variable
    stetig differenzierbar ist. Dann ist die charaktistische Funktion $x(t; t_0, x_0)$ stetig differenzierbar in
    $(t_0, x_0) \in T$ und $t \in I_{max}(t_0, x_0)$.
\end{satz}
$Beweis.$ \cite[69,70]{beckGewohnlicheDifferentialgleichungen2018}\\
\begin{bem}
    Man kann zeigen, dass für $T \subset \mathbb{R}^{1+n}$ und eine stetige Funktion $f:T \rightarrow \mathbb{R}^{n}$,
    die stetig differenzierbar in der x-Variable ist, gilt: $f$ ist lokal Lipschitz-stetig in der x-Variable.
    Das bedeutet, dass in Satz \ref{Satz-diff-abh} eine verstärkte Vorraussetzung an die rechte Seite
    im Gegensatz zu Satz \ref{Satz-stet-abh} verlangt wird.
\end{bem}