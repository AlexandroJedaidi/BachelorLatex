\section{Resultat}
\label{sec:resultat}
Wobei das neuronalen Netzes wie in Abschnitt \ref{subsec:rebound-Pendel} einen
nennenswerten Vorteil hat. Sobald das neuronale Netz trainiert ist, kann dieses zur Approximation anderer
Anfangswertprobleme, dessen Parameter in den Intervallen der Lösungspaketen liegen, genutzt werden.
dass unter Berücksichtigung des gegebenen Fehlers in Abbildung \ref{fig:stiff-error-in-time}, eine beliebige
Zeitauswertung in $[t_0, t_f]$ der approximierten Lösung möglich ist. Erhöht man die Toleranzen des BDF-Verfahrens, so
führt dies zu erhöhten Laufzeiten, was wiederum für eine Zeitauswertung in Summe längere Rechenzeiten haben könnte als
das bereits trainierte neuronale Netz.