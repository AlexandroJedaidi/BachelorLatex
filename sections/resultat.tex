\section{Resultat}
\label{sec:resultat}
Wir haben an verschiedenen Anfangswertproblemen gesehen, wie sich Trajektorien, Graphen in Abhängigkeit der Zeit,
Kostenfunktionen und weitere Fehler verhalten. Zusammenfassend können wir sagen, dass unter Berücksichtigung der
Hyperparameter und Trainingszeit der neuronalen Netze die Ergebnisse der numerischen Verfahren, namentlich dem
Runge-Kutta-Verfahren und dem BDF-Verfahren, den Approximationen der neuronalen Netze weit überlegen sind. Sofern wir
eine exakte Lösung gegeben haben, können wir die globalen Fehler beider Verfahren vergleichen und sehen, dass der
numerische Fehler um einige Größenordnungen kleiner ist als der Fehler des neuronalen Netzes.
Auch ohne exakte Lösung können wir unter anderem die Trajektorien der berechneten Lösungen vergleichen, wobei auch hier
die Trajektorie des Runge-Kutta-Verfahrens viel näher an dem Vergleichswert liegt, als die des neuronalen Netzes.
Außerdem sehen wir, dass die Wahl der Schichtanzahl und Größe abhängig von der Komplexität und Form des gegebenen
Systems ist. Deshalb ist die Wahl der Hyperparameter nicht trivial und wird hauptsächlich durch Ausprobieren ermittelt.\\
Der Fokus liegt also auf dem Trainingsprozess der neuronalen Netze, denn sofern dieser optimiert wird, gibt einen
nennenswerten Vorteil für das Approximieren durch neuronale Netze. Dieses kann zur Approximation anderer
Anfangswertprobleme, dessen Parameter in den Intervallen der Lösungspakete liegen, genutzt werden, da die Gewichte
durch die zufällige Wahl der Trainingsdaten auf viele Kombinationen von Anfangswertproblemen angepasst wurden. Ein
numerisches Lösen aller dieser Kombinationen kann je nach vorausgesetzter Genauigkeit zu größerem Rechenaufwand führen,
als die Auswertung des neuronalen Netzes, das durch eine Verkettung von Matrix-Vektor-Multiplikationen dargestellt ist.
Des Weiteren ist es möglich, dass unter Berücksichtigung des gegebenen Fehlers eines neuronalen Netzes, eine beliebige
Zeitauswertung $t_i \in [t_0, t_f]$ der approximierten Lösung möglich ist. Im Gegenzug dazu ist eine Zeitauswertung für
numerische Verfahren nicht möglich, da für den gesuchten Wert $u_i$ die Werte $ u_0, \dots, u_{i-1}$ zuerst berechnet
werden. Je nach vorausgesetzter Genauigkeit, Größe des Zeitintervalls und verwendetes numerisches Verfahren kann diese
Berechnung große Rechenaufwände aufweisen, während das neuronale Netz lediglich Matrix-Vektor-Multiplikationen
durchführt. Der geringe Rechenaufwand hat den Vorteil, dass zur Berechnung hochkomplexe, aufwendige Anfangswertprobleme
mit Computern ohne hohe Rechenleistung, also auch kleine Prozessoren die beispielsweise in Drohnen verbaut werden,
approximiert werden können \cite[9]{flamantSolvingDifferentialEquations2020}.\\
In Zukunft kann die Methode der Lösungspakete also Anwendungsbereiche finden, da mit steigender Rechenleistung sich auch
die Ergebnisse der Approximationen verbessern, welche dann wiederum gespeichert, verbreitet und für ähnliche
Anfangswertprobleme verwendet werden können.
